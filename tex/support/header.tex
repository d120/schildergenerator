
\usepackage[size=a4]{beamerposter}
\usepackage[utf8]{inputenc}
\usepackage{ngerman}
\usepackage{mathtools}

\usetheme{default}

%\usecolortheme{albatross}
%\usecolortheme{beaver}
%\usecolortheme{beetle}
%\usecolortheme{crane}
%\usecolortheme{dolphin}
%\usecolortheme{dove}
%\usecolortheme{fly}
%\usecolortheme{lily}
%\usecolortheme{orchid}
%\usecolortheme{rose}
\usecolortheme{seagull}
%\usecolortheme{seahorse}
%\usecolortheme{whale}
%\usecolortheme{wolverine}

\setbeamertemplate{headline}{} 

\setbeamertemplate{footline}{\hskip.5cm 
Schildergenerator Beispielschild\hfill
\includegraphics[width=.2\textwidth]{support/genericlogo}
\hskip.5cm~
\vskip.5cm
}

\setbeamertemplate{navigation symbols}{} % To remove the navigation symbols from the bottom of all slides uncomment this line

\usepackage{marvosym}

\usepackage{tikz}
\usetikzlibrary{shapes.arrows}

\usepackage{graphicx} % Allows including images
\usepackage{booktabs} % Allows the use of \toprule, \midrule and \bottomrulein tables

\newcommand{\defaulthead}[1]{%
	{\fontsize{120}{160}\selectfont #1}
}

\tikzset{
    myarrow/.style={
        draw,
        fill=black,
        single arrow,
        minimum width=.95\linewidth,
        minimum height=.95\linewidth,
        single arrow head extend=1ex
    }
}
\newcommand{\arrowup}{%
\tikz [baseline=14ex]{\node [myarrow,rotate=90] {};}
}
\newcommand{\arrowupright}{%
\tikz [baseline=17ex]{\node [myarrow,rotate=45] {};}
}
\newcommand{\arrowupleft}{%
\tikz [baseline=17ex]{\node [myarrow,rotate=135] {};}
}
\newcommand{\arrowdown}{%
\tikz [baseline=14ex]{\node [myarrow,rotate=-90] {};}
}
\newcommand{\arrowdownright}{%
\tikz [baseline=12ex]{\node [myarrow,rotate=-45] {};}
}
\newcommand{\arrowdownleft}{%
\tikz [baseline=12ex]{\node [myarrow,rotate=-135] {};}
}
\newcommand{\arrowright}{%
\tikz [baseline=14ex]{\node [myarrow] {};}
}
\newcommand{\arrowleft}{%
\tikz [baseline=14ex]{\node [myarrow,rotate=180] {};}
}
\newcommand{\arrowupthenleft}{%
  \tikz{\path[x=4mm, y=4mm, fill=black, draw=black, line width=0mm]
    (0,0) -- ++(0,8) to[out=90,in=0] ++(-12,12) -- ++(-4,0) -- ++(0,8)
    -- ++(-14,-14) -- ++(14,-14) -- ++(0,8) -| (-12,0) -- (0,0);}}
\newcommand{\arrowupthenright}{%
  \tikz{\path[x=4mm, y=4mm, fill=black, draw=black, line width=0mm]
    (0,0) -- ++(0,8) to[out=90,in=180] ++(12,12) -- ++(4,0) -- ++(0,8)
    -- ++(14,-14) -- ++(-14,-14) -- ++(0,8) -| (12,0) -- (0,0);}}
